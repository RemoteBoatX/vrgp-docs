As we have a big team, everyone has a specific role and certain responsibilities matching their skills, experience, and interests. We decided on the following roles and preliminary distribution of tasks:

\begin{itemize}
	\item \textbf{Project manager: Hredoy Mesha} \\
		The project manager is, in accordance with the definition in this project course, in charge of organizing the collaboration within the team, including distribution of tasks, keeping an overview of the project progress and deadlines, and team building events.
		\begin{itemize}
			\item Meeting organization and moderation
			\item Make sure everyone is following the deadlines
			\item Keep an overview of the project status
			\item Task supervision
			\item In charge of team happiness (organizing team events)
		\end{itemize}
	\item \textbf{Product owner: Alexandru Gherghescu} \\
		As defined in the course and the Scrum framework, the product owner is responsible for the software product and keeps an overview of the different, possibly parallel software design and development tasks.
		\begin{itemize}
			\item Make sure implementation respects the requirements
			\item Manage the design and architecture of the system
			\item Define, prioritize and adjust implementation-related tasks
			\item Manage the backlog of the project
		\end{itemize}
	\item \textbf{Scrum master: Gabriela Corbalan Gomez} \\
		The Scrum master, another role from the Scrum framework, is in charge of planning the sprints, creating and maintaining the backlog of tasks, and distributing the concrete tasks among the developers.
		\begin{itemize}
			\item Scrum toolchain and workflow
			\item Coach team members
			\item Assist product owner with the backlog of the project
		\end{itemize}
	\item \textbf{Quality manager: Daniel Gonzálvez Alfert} \\
		The quality manager ensures the quality of the software product, in the sense that the code itself is of high quality, by implementing appropriate quality assurance tools.
		\begin{itemize}
			\item Testing toolchain
			\item Make sure code meets quality requirements
			\item Keep track of error logs (problems that arise and solutions to them)
			\item Make sure developers follow the quality standards
			\item Performance checks
		\end{itemize}
	\item \textbf{Security manager: Joan Dolz Mensua} \\
		As we are working on a possibly, and at least theoretically, safety-critical software, the security manager is responsible for the software to be secure and safe.
		\begin{itemize}
			\item Make sure external software interaction is secure
			\item Make sure the implementation on top of the existing software stack is implemented correctly (from a security standpoint)
		\end{itemize}
	\item \textbf{Technical manager: Elijah Rose} \\
		The technical manager ensures a good workflow for the team by setting up and maintaining several tools for communication and collaboration including work on the documentation as well as the actual software.
		\begin{itemize}
			\item Maintain coding tools coherency
			\item Make sure people respect the conventions adopted in the tools used by the team
			\item Head developer
			\item Coding tasks distribution
		\end{itemize}
	\item \textbf{Documentation manager: Yannick Zapfe} \\
		The documentation manager is in charge of the documentation of the software product, starting with the requirements documentation and including the design documentation and finally the software documentation and the user guide.
		\begin{itemize}
			\item Communication with the client
			\item Make sure code is well documented and documentation is up-to-date
			\item Make sure technical and user documentation is up-to-date
		\end{itemize}
	\item \textbf{Developers: Alexandru, Gabriela,  Daniel,  Joan, Yannick, and Elijah} \\
		Except for the project manager, each team member will, in addition, also be involved in the software development process as a developer or tester. As we are still planning the software, more specific tasks and developer roles such as backend and frontend developer will be assigned in the future, when we know more about the requirements and technologies in the project. As we will mostly work on one piece of software and not a typical backend-frontend architecture, the roles will probably not differ much, but everyone will have some parts of the software they are expert and responsible for.
\end{itemize}

\noindent
The project has two customers belonging to two different organizations:

\begin{itemize}
	\item \textbf{Robert Aarts (Aboa Mare)} \\
		Aboa Mare provides the protocol specification as well as a rudimentary implementation of and testing environment for the protocol.
	\item \textbf{Kai Jämsä (Åboat)} \\
		Åboat is the target environment for the software product to work in. Kai Jämsä from Åboat provides the project with the technical necessities and knowledge for us to build a working software product.
\end{itemize}