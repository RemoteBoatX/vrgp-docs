\begin{enumerate}[a.]
	\item Co-development of protocol and implementation. The VRGP protocol is, as of now, unfinished and will be extended and refined at the same time that we develop the VRGP implementation. Therefore, we will need to pay attention to keeping the protocol specification and implementation in sync.
	\item Work distribution. Even being rather a big team, the job to do is not only difficult, but it implies the contribution to the VRGP specification, so every developer will have to be aware of the protocol evolution even if they don’t participate directly in it. This can bring us overwork and misinformation.
	\item Deadlines. Considering that the team members are students and that carries extra work such as studies, sports, and other activities, sometimes it is difficult to guarantee that the deliveries fit into the deadlines. This risk is even bigger taking into account the previous risk, which can make us lose time.
	\item Dependency. This project is just a subproject of a bigger one, Åboat, so it depends on the physical installation in the boat and of other services.
\end{enumerate}

\noindent
In order to mitigate risks a and b, the wisest decision is to start slowly and make everyone participate in the development of the protocol. Doing this we avoid starting to implement the VRGP protocol without a solid specification and at the same time assure us that everyone knows the protocol. The disadvantage of this approach is the tight schedule it leaves for implementing the protocol.
\\\\
Risk c is half-mitigated with the previous approach to the development of the protocol implementation. There’s not much to do in order to gain time as we are students and will be during the whole time. We can only prevent it by having a good organization and a well-structured schedule.
\\\\
For risk d, in case we can’t test with the physical boat or other pieces of the architecture are unavailable, we can develop small simulation codes.